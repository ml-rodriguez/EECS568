\documentclass[12pt]{article}
\usepackage[pdftex]{graphicx}
\usepackage{epstopdf}
\usepackage{amsmath, algorithmic, color, multicol}
\usepackage{subfigure}


\title{Answers to Problem Set 3}
\author{
	Lauren Hinkle (lhinkle)\\
	Pedro d'Aquino (pdaquino)\\
	Robert Goeddel (rgoeddel)}

\begin{document}
\maketitle
\pagebreak

%% TASK 1
\section{FastSLAM}

% PART A
\paragraph{A}
Starting with an existing EKF...
\subparagraph{i}
After first observation update, you need to grow the covariance matrix to
be $7 \times 7$, but we don't expect the covariances of the existing matrix
to change. We do \emph{not} apply the normal Kalman gain equations. Instead,
we add rows to $J_x^f$. % XXX What is f?
In this case, we have observations $(r_1, \theta_1)$ that can be projected
into $(x,y)$ space from position $(x_0, y_0, \theta_0)$.
XXX % XXX WARNING, stuff here needs to be written

$K = \Sigma_x J_x^T(J_x \Sigma_x J_x^T + \Sigma_z)^{-1} $ \\
$\Sigma_z = \left[ \begin{array}{c c}
3 & 0 \\
0 & 1 \\
\end{array} \right]$ \\
$x' = x + Kr$ \\
$\Sigma_x' = (I - KJ_x) \Sigma_x$ \\

\subparagraph{ii}
After second observation update (\emph{Note: additional primes denote
    later state}).
$K' = \Sigma_x' J_{x'}^T(J_{x'} \Sigma_x' J_{x'}^T + \Sigma_z)^{-1} $ \\
$x'' = x' + K'r'$ \\
$\Sigma_x'' = (I - K'J_{x'}) \Sigma_x'$ \\

\subparagraph{iii}
After observing $f_1$: \\
$$\mu = \left[ \begin{array}{c} 
3 \\
2 \\
pi \\
12 \\
15 \\
3 \\
-8
\end{array}\right],
\Sigma = \left[ \begin{array}{c c c c c c c} 
4.00 & 1.00 & 2.00  & 1.00 & 1.00 & 1.00 & -2.40 \\
1.00 & 6.00 & 3.00 & 1.00 & 2.00 & 6.00 & -3.10 \\
2.00 &  3.00 & 4.00 & 1.00 & 2.00 & 3.00 & -4.20 \\
1.00 & 1.00 & 1.00 & 8.00 & 1.00 & 1.00 & -1.10 \\
1.00 & 2.00 & 2.00 & 1.00 & 10.00 & 2.00 & -2.10 \\ 
1.00 & 6.00 & 3.00 & 1.00 & 2.00 & 106.00 & -3.10 \\
-2.40 & -3.10 & -4.20 & -1.10 & -2.10 & -3.10 & 7.44 
\end{array}\right]$$

after re-observing $f_1$: 

$$\mu = \left[ \begin{array}{c}
2.95\\
2.43\\
3.18\\
12.03\\
15.07\\
4.97\\
-8.30\\
\end{array}\right],
\Sigma = \left[ \begin{array}{c c c c c c c} 
-0.94 & -1.31 & -1.67 & -0.44 & -0.85 & 2.00 & 1.93 \\
-1.31& -3.66 & -2.82 & -0.83 & -1.62 & -4.08 & 4.19 \\
-1.67 & -2.82 & -3.10 & -0.84 & -1.62 & 1.73 & 3.83 \\
-0.44 & -0.83 & -0.84 & -0.23 & -0.45 & 0.15 & 1.08 \\
-0.85 & -1.62 & -1.62 & -0.45 & -0.87 & 0.16 & 2.09 \\
2.00 & -4.08 & 1.73 & 0.15 & 0.16 & -29.89 & 1.54 \\
1.93 & 4.19 & 3.83 & 1.08 & 2.09 & 1.54 & -5.18\\
\end{array}\right]$$

\subparagraph{iv} Compute the mean and covariance of $f_1$:
$$\mu = \left[ \begin{array}{c}
x_0 + rcos(\theta_0 + \phi) \\
y_0 + rsin(\theta_0 + \phi) \\
\end{array}\right]$$
$$J =  \left[ \begin{array}{c c}
cos(\theta_0 + \phi) & -rsin(\theta_0 + \phi)\\
sin(\theta_0 + \phi) & rcos(\theta_0 + \phi)\\
\end{array}\right]$$

\subparagraph{v} Update the landmark mean and covariance given re-observation of $f_1$:
$$K = \Sigma_0J^T\left(J\Sigma_0J^T + \Sigma_w\right)^{-1}$$
$$\mu_1 = \mu_0 + Kr$$
$$\Sigma_1 = (I-KJ)\Sigma_0$$
where $\mu_0$ and $\Sigma_0$ are the mean and covariance before this observation and $\mu_1$ and $\Sigma_1$ are the new mean and covariance.

\subparagraph{vi}
After observing $f_1$: \\
$$\mu_0 = \left[ \begin{array}{c} 
3 \\
-8
\end{array}\right],
\Sigma_0 = \left[ \begin{array}{c c} 
100 & 0 \\
0 & 3
\end{array}\right]$$

after re-observing $f_1$: 
$$\mu_1 = \left[ \begin{array}{c}
4.57 \\
-8.5
\end{array}\right],
\Sigma_1 = \left[ \begin{array}{c c c c c c c} 
50 & 0 \\
0 & 1.5
\end{array}\right]$$

% Task 1 part B-D
\paragraph{B}

\paragraph{C}

\paragraph{D}

\paragraph{E}

\paragraph{F}

%% Task 2
\section{Line Estimation from Laser Data}

\paragraph{A}

\paragraph{B}

\paragraph{C}

%% Task 3
\section{RANSAC Rigid-Body Transformation}

\paragraph{A}

\paragraph{B}

\paragraph{C}

%% Task 4
\section{Advanced Data Association }

\end{document}
