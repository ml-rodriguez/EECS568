\documentclass[12pt]{article}
\usepackage[pdftex]{graphicx}
\usepackage{epstopdf}
\usepackage{amsmath, algorithmic, color, multicol}
\usepackage{subfigure}


\title{Answers to Problem Set 3}
\author{
	Lauren Hinkle (lhinkle)\\
	Pedro d'Aquino (pdaquino)\\
	Robert Goeddel (rgoeddel)}

\begin{document}
\maketitle
\pagebreak

%% TASK 1
\section{FastSLAM}

% PART A
\paragraph{A}
Starting with an existing EKF...
\subparagraph{i}
After first observation of a new landmark, the covariance matrix grows from $5\times 5$ to
$7 \times 7$, but we don't expect the covariances of the existing matrix
to change. We do \emph{not} apply the normal Kalman gain equations. Instead,
we add rows to $J_x^f$. % XXX What is f?
In this case, we have observations $(r_1, \theta_1)$ that can be projected
into $(x,y)$ space from position $(x_0, y_0, \theta_0)$. % XXX WARNING, stuff here needs to be written

$$K = \Sigma_x J_x^T(J_x \Sigma_x J_x^T + \Sigma_z)^{-1} $$
$$\Sigma_z = \left[ \begin{array}{c c}
3 & 0 \\
0 & 1 \\
\end{array} \right]$$
$$x' = x + Kr$$
$$\Sigma_x' = (I - KJ_x) \Sigma_x$$

\subparagraph{ii}
After re-observing $f_1$, the mean and covariance matrix are calculated using an updated Kalman gain (\emph{Note: additional primes denote
    later state}).
$$K' = \Sigma_x' J_{x'}^T(J_{x'} \Sigma_x' J_{x'}^T + \Sigma_z)^{-1} $$
$$x'' = x' + K'r'$$
$$\Sigma_x'' = (I - K'J_{x'}) \Sigma_x'$$

\subparagraph{iii}
After observing $f_1$: \\
$$\mu = \left[ \begin{array}{c}
3 \\
2 \\
pi \\
12 \\
15 \\
3 \\
-8
\end{array}\right],
\Sigma = \left[ \begin{array}{c c c c c c c}
4.00 & 1.00 & 2.00 & 1.00 & 1.00 & 1.00 & -2.40 \\
1.00 & 6.00 & 3.00 & 1.00 & 2.00 & 6.00 & -3.10 \\
2.00 & 3.00 & 4.00 & 1.00 & 2.00 & 3.00 & -4.20 \\
1.00 & 1.00 & 1.00 & 8.00 & 1.00 & 1.00 & -1.10 \\
1.00 & 2.00 & 2.00 & 1.00 & 10.00 & 2.00 & -2.10 \\
1.00 & 6.00 & 3.00 & 1.00 & 2.00 & 106.00 & -3.10 \\
-2.40 & -3.10 & -4.20 & -1.10 & -2.10 & -3.10 & 7.44
\end{array}\right]$$

after re-observing $f_1$:

$$\mu = \left[ \begin{array}{c}
2.95\\
2.43\\
3.18\\
12.03\\
15.07\\
4.97\\
-8.30\\
\end{array}\right],
\Sigma = \left[ \begin{array}{c c c c c c c}
-0.94 & -1.31 & -1.67 & -0.44 & -0.85 & 2.00 & 1.93 \\
-1.31& -3.66 & -2.82 & -0.83 & -1.62 & -4.08 & 4.19 \\
-1.67 & -2.82 & -3.10 & -0.84 & -1.62 & 1.73 & 3.83 \\
-0.44 & -0.83 & -0.84 & -0.23 & -0.45 & 0.15 & 1.08 \\
-0.85 & -1.62 & -1.62 & -0.45 & -0.87 & 0.16 & 2.09 \\
2.00 & -4.08 & 1.73 & 0.15 & 0.16 & -29.89 & 1.54 \\
1.93 & 4.19 & 3.83 & 1.08 & 2.09 & 1.54 & -5.18\\
\end{array}\right]$$

\subparagraph{iv} Compute the mean and covariance of $f_1$:
$$\mu = \left[ \begin{array}{c}
x_0 + rcos(\theta_0 + \phi) \\
y_0 + rsin(\theta_0 + \phi) \\
\end{array}\right]$$
$$J =  \left[ \begin{array}{c c}
cos(\theta_0 + \phi) & -rsin(\theta_0 + \phi)\\
sin(\theta_0 + \phi) & rcos(\theta_0 + \phi)\\
\end{array}\right]$$

\subparagraph{v} Update the landmark mean and covariance given re-observation of $f_1$:
$$K = \Sigma_0J^T\left(J\Sigma_0J^T + \Sigma_w\right)^{-1}$$
$$\mu_1 = \mu_0 + Kr$$
$$\Sigma_1 = (I-KJ)\Sigma_0$$
where $\mu_0$ and $\Sigma_0$ are the mean and covariance before this observation and $\mu_1$ and $\Sigma_1$ are the new mean and covariance.

\subparagraph{vi}
After observing $f_1$: \\
$$\mu_0 = \left[ \begin{array}{c}
3 \\
-8
\end{array}\right],
\Sigma_0 = \left[ \begin{array}{c c}
100 & 0 \\
0 & 3
\end{array}\right]$$

after re-observing $f_1$:
$$\mu_1 = \left[ \begin{array}{c}
4.57 \\
-8.5
\end{array}\right],
\Sigma_1 = \left[ \begin{array}{c c c c c c c}
50 & 0 \\
0 & 1.5
\end{array}\right]$$

% Task 1 part B-D
\paragraph{B}
FastSLAM 1.0 samples new robot positions from a distrubtion taking only
odometry measurements into account. However, it is often the case that
a robot's sensors are more accurate than its odometry and control. FastSLAM 2.0
leverages these accurate observation measurements to its advantage, sampling
new poses based on odometry \emph{and} observation measurements. This prevents
FastSLAM 2.0 from sampling as many low likelihood particles, making it more
efficient.

\paragraph{C}
Our particle resampling method resamples particles whenever a new observation
is made, since this is when the weights change. We iterate through all
particles to sum their weights, and then we draw random values between 0 and
the summed weight to select new particles. To determine which particle to pick,
we iterate through the particles, summing the weights again as we go, and when
the sum is greater than our random value, we return a copy of that particle.

To test for sampling bias, we can easily just sample an excessive amount of
particles and compare the resulting distribution of particles to our
expected distribution. If we sample many, many particles, large differences
in the resulting set of particles compared to our expected set will become
increasingly improbable and thus detectable with a certain confidence.

\paragraph{D}
Since every particle is an independent entity, particles that make bad
associations will be killed off during resampling. Therefore, our
expectation is that a particle-based method like FastSLAM can get away
with a fairly poor data association technique like nearest-neighbor
because only good associations will survive. A method like least-squares
SLAM commits to the poor associations it makes and may never recover, so
data association techniques must be ``smarter'' than nearest-neighbor
matching.

\paragraph{E}
Attached in email.

%XXX Make visualization key

\paragraph{F}
FastSLAM is easy to implement and does not require very ``intelligent'' data
association techniques to work well. In fact, a big advantage of FastSLAM is
that it is more robust to poor data associations than alternative SLAM
algorithms. Least-squares may run faster than FastSLAM, given the potentially
large particle requirements of FastSLAM, and it is more consistent in that
there is no random element to the problem. Least-squares also maintains
complete information forever, unlike FastSLAM, so it will not throw away
potentially useful information as FastSLAM does in cases involving particle
depletion. In other words, least-squares is good at loop closures.

FastSLAM is a safe choice in environments where data association is difficult.
If one can afford enough particles, it is probable that a particle that makes
the correct associations and can survive to give a good map. Least-squares SLAM
would likely not recover from a bad association without interference. However,
in cases where we are confident in our data association, methods like
least-squares SLAM converge faster due to the preservation of loops and, given
the optimizations employed in the past few years, are otherwise comparable to
FastSLAM in terms of speed of computation.

SLAM research has likely left particle filters behind because, given the option,
it is preferable to represent the entire problem, that is, to not throw out
constraint information, rather than rely on our best solutions to survive
filtering. %XXX More could be done here

%% Task 2
\section{Line Estimation from Laser Data}

\paragraph{A}Deriving a closed-form expression that computes the MSE:
Given
$$M_x = \displaystyle\sum_i x_i \qquad M_{xx} = \displaystyle\sum_i x_i^2 \qquad q_x = \frac{M_x}{N}$$
$$M_y = \displaystyle\sum_i y_i \qquad M_{yy} = \displaystyle\sum_i y_i^2 \qquad q_y = \frac{M_y}{N}$$
 $$\hat{n} = \left[ \begin{array}{c}
-sin\theta \\
cos\theta
\end{array}\right]$$

\begin{align*} MSE &=  \displaystyle\sum_i \left[\left(x_i - q_x)\hat{n_x} + (y_i - q_y)\hat{n_y}\right)\right]^2\\
&=\displaystyle\sum_i \left[-\left(x_i - \frac{M_x}{N}\right)sin\theta + \left(y_i - \frac{M_y}{N}\right)cos\theta\right]^2 \\
&=\displaystyle\sum_i \left(\left(y_i - \frac{M_y}{N}\right)cos^2\theta \right)
-2\displaystyle\sum_i \left(\left(x_i-\frac{M_x}{N}\right)\left(y_i - \frac{M_y}{N}\right)cos\theta sin\theta\right) \\
 &\qquad+ \displaystyle\sum_i \left(\left(x_i - \frac{M_x}{N}\right)cos^2\right) \\
&= cos^2\theta(M_{yy}-\frac{M_y^2}{N}) - 2cos\theta sin\theta(M_{xy}-\frac{M_xM_y}{N}) + sin^2\theta(M_{xx}-\frac{M_x^2}{N})
\end{align*}

\paragraph{B}
The error threshold controls how large an error is acceptable when merging 
two lines, or how close the original lines were to being the same line.  If the 
MSE of the line created by joining the points of two lines is less than the threshold, 
the points are assumed to be close enough to a straight line, and that any error
is due to noise in $r$ or $\theta$.  An error in $\theta$ will cause us to see points 
that don't exist, but places those points close to an existing line.  Therefore, errors in
$\theta$ are less catastrophic than errors in $r$.  We can use this to determine
a good error threshold.

The variance in $r$ in this log is $.001$, an error within one standard deviation
of $r$ must be less than approximately $.03$, the number we selected as our 
default parameter for the error threshold.  This returns reasonable results (as seen
in the accompanying figures).

\paragraph{C}
The idea is to weight fitting errors according to the probability that they came from sensor noise, as opposed to a bad fit. To do that, we project the covariance of the observation from $(r,\theta)$ space to $(x,y)$ and use its inverse as a weight factor in the MSE.

Let $p_i$ be an observation and $q$ a point on the line:

\[MSE = \sum{\left((p_i-q)^T\left(J\Sigma_zJ^T\right)^{-1}\hat{n}\right)^2} \]

Where $\hat{n}$ is a vector normal to the line, $\Sigma_z$ is the observation covariance in $(r,\theta)$ and $J$ is the jacobian of the function that maps observations in $(r,\theta)$ to the global $(x,y)$ coordinate frame.

This more principled model would make a significant difference when there is a lot of noise.  In this case the original MSE formula might not allow for points to be combined into a line because they are outside a given threshold.  This new calculation for MSE would weight down the seemingly large errors by taking into account that the large difference is more likely. 
%% Task 3
\section{RANSAC Rigid-Body Transformation}

\paragraph{A}
Given a number of line correspondences, a set of 3 non-parallel lines that
intersect at unique locations would be sufficient to product a 2D RBT. To
compute the RBT, find the intersections of the lines in each set and then
use Horn's Algorithm to align the resulting sets of 3 points.

\paragraph{B}
Given a fit between lines in set A and lines in set B, we compute a RBT that
rotates A into B's space. Then, for every point in A that was used to estimate
the line fits, we find the line in set B that the point in A is closest to and
see if the distance between the A and line B is within a threshold. If so,
we vote $1$ for that point, else $0$. The consensus score is the sum of these
votes.

\paragraph{C}

%% Task 4
\section{Advanced Data Association }
	Data association is the attempt to correlating observed features when we do
not have known data association.  Without good data association, SLAM won't work.
The three ways we have discussed doing data association in class are $\chi^2$ 
nearest neighbor, RANSAC, and JCBB.

	$\chi^2$ nearest neighbor %XXXX
	
	RANSAC uses random search to choose a model for associating  
RANSAC is easy to implement and doesn't require points to be in any order since
it randomly selects them (however this means that it doesn't take advantage of points
that are ordered).

	JCBB seeks to create as few new features as possible (associate as many features
as it can) without surpassing a threshold.  It does this by intelligently searching a tree of
possible associations in which a newly observed feature can be associated with any
previously observed feature, and if none of these produce a $\chi^2$ less than a given 
threshold the observation becomes a new feature.  It 

The advantages of JCBB
\end{document}
